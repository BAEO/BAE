\section{Detector and simulation}
\label{sec:Detector}
The paragraph below can be used for the detector
description. Modifications may be required in specific papers to fit
within page limits, to enhance particular detector elements or to
introduce acronyms used later in the text. For journals where strict
word counts are applied (for example, PRL), and space is at a premium,
it may be sufficient to write, as a minimum: ``The LHCb detector is a 
single-arm forward spectrometer covering the pseudorapidity range 
$2 < \eta < 5$, 
described in detail in Refs.~\cite{Alves:2008zz,LHCb-DP-2014-002}''. 
A slightly longer version could specify the most relevant sub-detectors, {\it e.g} 
``The LHCb 
detector~\cite{Alves:2008zz,LHCb-DP-2014-002} is a
single-arm forward spectrometer covering the pseudorapidity range $2 < \eta < 5$, designed for
the study of particles containing b or c quarks. The detector elements that are particularly
relevant to this analysis are: a silicon-strip vertex detector surrounding the pp interaction
region that allows c- and b-hadrons to be identified from their characteristically long
flight distance; a tracking system that provides a measurement of momentum, $p$, of charged
particles; and two ring-imaging Cherenkov detectors that are able to discriminate between
different species of charged hadrons.'' 

\begin{verbatim}
In the following paragraph, references to the individual detector 
performance papers are marked with a * and should only be included 
if the analysis relies on numbers or methods described in the specific 
papers. Otherwise, a reference to the overall detector performance 
paper~\cite{LHCb-DP-2014-002} will suffice. Note also that the text 
defines the acronyms for primary vertex, PV, and impact parameter, IP. 
Remove either of those in case it is not used later on.
\end{verbatim}

The \lhcb detector~\cite{Alves:2008zz,LHCb-DP-2014-002} is a single-arm forward
spectrometer covering the \mbox{pseudorapidity} range $2<\eta <5$,
designed for the study of particles containing \bquark or \cquark
quarks. The detector includes a high-precision tracking system
consisting of a silicon-strip vertex detector surrounding the $pp$
interaction region~\cite{LHCb-DP-2014-001}\verb!*!, a large-area silicon-strip detector located
upstream of a dipole magnet with a bending power of about
$4{\mathrm{\,Tm}}$, and three stations of silicon-strip detectors and straw
drift tubes~\cite{LHCb-DP-2013-003}\verb!*! placed downstream of the magnet.
The tracking system provides a measurement of momentum, \ptot, of charged particles with
a relative uncertainty that varies from 0.5\% at low momentum to 1.0\% at 200\gevc.
The minimum distance of a track to a primary vertex (PV), the impact parameter (IP), 
is measured with a resolution of $(15+29/\pt)\mum$,
where \pt is the component of the momentum transverse to the beam, in\,\gevc.
Different types of charged hadrons are distinguished using information
from two ring-imaging Cherenkov detectors~\cite{LHCb-DP-2012-003}\verb!*!. 
Photons, electrons and hadrons are identified by a calorimeter system consisting of
scintillating-pad and preshower detectors, an electromagnetic
calorimeter and a hadronic calorimeter. Muons are identified by a
system composed of alternating layers of iron and multiwire
proportional chambers~\cite{LHCb-DP-2012-002}\verb!*!.
The online event selection is performed by a trigger~\cite{LHCb-DP-2012-004}\verb!*!, 
which consists of a hardware stage, based on information from the calorimeter and muon
systems, followed by a software stage, which applies a full event
reconstruction.

A more detailed description of the 'full event reconstruction' could be:
\begin{itemize}
\item The trigger~\cite{LHCb-DP-2012-004}\verb!*! consists of a
hardware stage, based on information from the calorimeter and muon
systems, followed by a software stage, in which all charged particles
with $\pt>500\,(300)\mev$ are reconstructed for 2011\,(2012) data.
For triggers that require neutral particles, 
energy deposits in the electromagnetic calorimeter are 
analysed to reconstruct \piz and $\gamma$ candidates.
\end{itemize}

The trigger description has to be specific for the analysis in
question. In general, you should not attempt to describe the full
trigger system. Below are a few variations that inspiration can be
taken from. First from a hadronic analysis, and second from an
analysis with muons in the final state. In case you have to look 
up specifics of a certain trigger, a detailed description of the trigger 
conditions for Run 1 is available in Ref.~\cite{LHCb-PUB-2014-046}. 
{\bf Never cite this note in a PAPER or CONF-note.} 


\begin{itemize}
\item At the hardware trigger stage, events are required to have a muon with high \pt or a
  hadron, photon or electron with high transverse energy in the calorimeters. For hadrons,
  the transverse energy threshold is 3.5\gev.
  The software trigger requires a two-, three- or four-track
  secondary vertex with a significant displacement from the primary
  $pp$ interaction vertices. At least one charged particle
  must have a transverse momentum $\pt > 1.7\gevc$ and be
  inconsistent with originating from a PV.
  A multivariate algorithm~\cite{BBDT} is used for
  the identification of secondary vertices consistent with the decay
  of a \bquark hadron.
%\item The software trigger requires a two-, three- or four-track
%  secondary vertex with a large sum of the transverse momentum, \pt, of
%  the tracks and a significant displacement from the primary $pp$
%  interaction vertices~(PVs). At least one track should have $\pt >
%  1.7\gevc$ and \chisqip with respect to any
%  primary interaction greater than 16, where \chisqip is defined as the
%  difference in \chisq of a given PV reconstructed with and
%  without the considered track.\footnote{If this sentence is used to define \chisqip
%  for a composite particle instead of for a single track, replace ``track'' by ``particle'' or ``candidate''}
% A multivariate algorithm~\cite{BBDT} is used for
%  the identification of secondary vertices consistent with the decay
%  of a \bquark hadron.
\item The $\decay{\Bd}{\Kstarz\mumu}$ signal candidates are first required
      to pass the hardware trigger, which selects events containing at least
      one muon with transverse momentum $\pt>1.48\gevc$ in the 7\tev data or
      $\pt>1.76\gevc$ in the 8\tev data.  In the subsequent software
      trigger, at least one of the final-state particles is required to 
      have $\pt>1.7\gevc$ in the 7\tev data or $\pt>1.6\gevc$ in the 8\tev 
      data, unless the particle is identified as a muon in which case 
      $\pt>1.0\gevc$ is required. The final-state particles that 
      satisfy these transverse momentum criteria are also required 
      to have an impact parameter larger than $100\mum$ with respect 
      to all PVs in the event. Finally, the tracks of two or more of 
      the final-state particles are required to form a vertex that is 
      significantly displaced from the PVs." 

%  Candidate events are first required to pass the hardware trigger,
%  which selects muons with a transverse momentum $\pt>1.48\gevc$ 
%  in the 7\tev data or $\pt>1.76\gevc$ in the 8\tev data.
%  In the subsequent software trigger, at least
%  one of the final-state particles is required to have both
%  $\pt>0.8\gevc$ and impact parameter larger than $100\mum$ with respect to all
%  of the primary $pp$ interaction vertices~(PVs) in the
%  event. Finally, the tracks of two or more of the final-state
%  particles are required to form a vertex that is significantly
%  displaced from the PVs.
\end{itemize}

For analyses using the 2015 Turbo stream, the following paragraph may 
be used to describe the trigger.
\begin{itemize}
\item The online event selection is performed by a trigger. This consists 
      of a hardware stage, which, for this analysis, randomly selects a 
      pre-defined fraction of all beam-beam crossings at a rate of 300 kHz, 
      followed by a software stage. In between the hardware and software
      stages, an alignment and calibration of the detector is performed in 
      near real-time \cite{LHCb-PROC-2015-011} and updated constants are made
      available for the 
      trigger. The same alignment and calibration information is propagated 
      to the offline reconstruction, ensuring consistent and high-quality 
      particle identification (PID) information between the trigger and 
      offline software. The identical performance of the online and offline 
      reconstruction offers the opportunity to perform physics analyses 
      directly using candidates reconstructed in the trigger 
      \cite{LHCb-DP-2012-004,LHCb-DP-2016-001} 
      which the present analysis exploits. The storage of only the triggered
      candidates enables a reduction in the event size by an order 
      of magnitude.
\end{itemize}

An example to describe the use of both TOS and TIS events:
\begin{itemize}
\item In the offline selection, trigger signals are associated with reconstructed particles.
%Selection requirements can therefore be made not only on the trigger requirement,
%but on whether the decision was due to the signal candidate, other particles produced in the $pp$ collision, or a combination of both.
Selection requirements can therefore be made on the trigger selection itself
and on whether the decision was due to the signal candidate, other particles produced in the $pp$ collision, or a combination of both.
\end{itemize}

A good example of a description of long and downstream \KS is given in 
Ref.~\cite{LHCb-PAPER-2014-006}:
\begin{itemize}
\item
Decays of \decay{\KS}{\pip\pim} are reconstructed in two different categories:
the first involving \KS mesons that decay early enough for the
daughter pions to be reconstructed in the vertex detector; and the
second containing \KS that decay later such that track segments of the
pions cannot be formed in the vertex detector. These categories are
referred to as \emph{long} and \emph{downstream}, respectively. The
long category has better mass, momentum and vertex resolution than the
downstream category.
\end{itemize}

The description of our software stack for simulation is often
causing trouble. The following paragraph can act as inspiration but
with variations according to the level of detail required and if
mentioning of \eg \photos is required.
\begin{itemize}
\item In the simulation, $pp$ collisions are generated using
\pythia~\cite{Sjostrand:2006za,*Sjostrand:2007gs} 
(In case only \pythia 6 is used, remove \verb=*Sjostrand:2007gs= from this citation; if 
only \pythia 8 is used, then reverse the order of the papers in the citation.)
 with a specific \lhcb
configuration~\cite{LHCb-PROC-2010-056}.  Decays of hadronic particles
are described by \evtgen~\cite{Lange:2001uf}, in which final-state
radiation is generated using \photos~\cite{Golonka:2005pn}. The
interaction of the generated particles with the detector, and its response,
are implemented using the \geant
toolkit~\cite{Allison:2006ve, *Agostinelli:2002hh} as described in
Ref.~\cite{LHCb-PROC-2011-006}.
\end{itemize}

A quantity often used in LHCb analyses is \chisqip. When mentioning it in 
a paper, the following wording could be used: ``$\ldots$\chisqip\ with respect 
to any primary interaction vertex greater than X, where \chisqip\ is defined as 
the difference in the vertex-fit \chisq of a given PV reconstructed with and
without the track under consideration/being considered.''\footnote{If this
sentence is used to define \chisqip\ for a composite particle instead of 
for a single track, replace ``track'' by ``particle'' or ``candidate''}

Many analyses depend on boosted decision trees. It is inappropriate to
use TMVA as the reference as that is merely an implementation of the
BDT algorithm. Rather it is suggested to write: ``In this paper we use a 
boosted decision tree~(BDT)~\cite{Breiman,AdaBoost} to separate signal 
from background''.

When describing the integrated luminosity of the data set, do not use
expressions like ``1.0\,fb$^{-1}$ of data'', but \eg 
``data corresponding to an integrated luminosity of 1.0\,fb$^{-1}$'', 
or ``data obtained from 3\,fb$^{-1}$ of integrated luminosity''. 

For analyses where the periodical reversal of the magnetic field is crucial, 
\eg in measurements of direct \CP violation, the following description can be
used as an example phrase: 
``The polarity of the dipole magnet is reversed periodically throughout data-taking.
The configuration with the magnetic field vertically upwards, \MagUp (downwards, \MagDown), bends positively (negatively)
charged particles in the horizontal plane towards the centre of the LHC.''
Only use the \MagUp, \MagDown symbols if they are used extensively in tables or figures.
